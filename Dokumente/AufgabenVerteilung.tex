\NeedsTeXFormat{LaTeX2e}
\documentclass[a4paper,12pt,
headsepline,           % Linie zw. Kopfzeile und Text
oneside,               % zweiseitig
pointlessnumbers,      % keine Punkte nach den letzten Ziffern in Überschriften
bibtotoc,              % LV im IV
%DIV=15,               % Satzspiegel auf 15er Raster, schmalere Ränder   
BCOR15mm               % Bindekorrektur
%,draft
]{scrbook}
\KOMAoptions{DIV=last} % Neuberechnung Satzspiegel nach Laden von Paket helvet

\pagestyle{headings}
\usepackage{blindtext}

% für Texte in deutscher Sprache
\usepackage[ngerman]{babel}
\usepackage[utf8]{inputenc}
\usepackage[T1]{fontenc}

% Helvetica als Standard-Dokumentschrift
\usepackage[scaled]{helvet}
%\usepackage{mathptmx}
\renewcommand{\familydefault}{\sfdefault} 

\usepackage{graphicx}

% Literaturverzeichnis mit BibLKaTeX
\usepackage[babel,german=quotes]{csquotes}
\usepackage[backend=bibtex8]{biblatex}
\bibliography{bibliography}

% Für Tabellen mit fester Gesamtbreite und variabler Spaltenbreite
\usepackage{tabularx} 

% Besondere Schriftauszeichnungen
\usepackage{url}              % \url{http://...} in Schreibmaschinenschrift
\usepackage{color}            % zum Setzen farbigen Textes

\usepackage{amssymb, amsmath} % Pakete für Mathe-Umgebungen und -Symbole

\usepackage{setspace}         % Paket für div. Abstände, z.B. ZA
%\onehalfspacing              % nur dann, wenn gefordert; ist sehr groß!!
\setlength{\parindent}{0pt}   % kein linker Einzug der ersten Absatzzeile
\setlength{\parskip}{1.4ex plus 0.35ex minus 0.3ex} % Absatzabstand, leicht variabel

% Tiefe, bis zu der Überschriften in das Inhaltsverzeichnis kommen
\setcounter{tocdepth}{3}      % ist Standard

% Beispiele für Quellcode
\usepackage{listings}
\lstset{language=Java,
  showstringspaces=false,
  frame=single,
  numbers=left,
  basicstyle=\ttfamily,
  numberstyle=\tiny}

%\usepackage[printonlyused, withpage]{acronym}
\usepackage[printonlyused]{acronym}
\usepackage{comment}
\usepackage{longtable}
%\usepackage[maxdepth=8, mark]{gitinfo2}
\usepackage{bytefield}
\usepackage{smartdiagram}
\usepackage{caption}
\usepackage{hyperref}
\usepackage{hypcap}
\usepackage{graphicx}
\usepackage{tikz}
\usepackage{mathtools}
\usepackage{colortbl}
\usepackage{xcolor}
\usepackage[outline]{contour}
\usepackage{varwidth}
\usepackage{array}

\makeatletter
\newcommand{\minipagetrue}{\@minipagetrue}

\makeatother

\usepackage{enumitem}
\setlist[itemize]{nosep,after=\vskip-\baselineskip,label={\textbullet},leftmargin=*,before=\minipagetrue,}

\newcolumntype{P}[1]{>{\raggedright\arraybackslash}p{#1}}

\begin{document}
  \section*{Projektvertilung}
  \begin{tabular}{@{}lP{6cm}@{}} 
      \textbf{Projektgebiet} & \textbf{Person} \\[5pt]
      App Backend &
        \begin{itemize}
          \item Björn 
          \item Babsi
          \item Jojo
        \end{itemize}\\

      App Design &
      \begin{itemize}
          \item Sebi
          \item Alex
          \item Fabian
      \end{itemize}\\

      Web Server &
      \begin{itemize}
          \item Döms
          \item Fabian
          \item Björn
      \end{itemize}\\
  \end{tabular}

  \section*{Aufgaben im Projekt}
  \subsection*{App Backend}
  \begin{itemize}
      \item Verbinden mit Web Server
      \item Verknüpfen einzelner Fragmente duch Buttons,\dots
      \item Hintergrundaktivität
  \end{itemize}

  \subsection*{App Design}
  \begin{itemize}
      \item Erstellen eines Styler für Fragmente, Texte,\dots
      \item Festlegen eines Farb Themas in der App
      \item Evtl Style via CSS generieren
      \item Einbinden des Logos oder Festivalnamen als Überschriften
  \end{itemize}

  \subsubsection*{Web Server}
  \begin{itemize}
      \item Aufsetzen eines Web Servers mittels Apache,\dots
      \item Anlegen der Daten für App in geeigneter Struktur
      \item Installieren und aufsetzen von RSS
  \end{itemize}

  \section*{TODO}
  \begin{itemize}
    \item Navigations Menü erstellen @Björn
    \item Fragmente mit leeren Views
  \end{itemize}

  \section*{Weitere Vorschläge}
  \begin{itemize}
    \item Interaktivität per App bei Sachen die Stören
    \item Voranmerkungen für Acts, per Push Benachrichtigungen
    \item Act Vorbemerkung mit Timer, der einen informaiert
    \item iframe Integration von der Website zur App
    \item Impressum und Datenschtz per Link
    \item Karte mit Treffpunkt zu Kontakten
    \item Kontakte, Ansprechpartner, Notruf, Führerbunker, Info Point
    \item Dorfladen mit in Essen Menü übernehmen
    \item Favoriten Markieren und merken jede Person
    \item Menüleiste unten für einzelne Fragmente hinzufügen
    \item News Anzeige mit laufenden Informationen
    \item Member Area mit Arbeitsplan, \dots
  \end{itemize}

\section*{Projektvertilung}
\begin{tabular}{@{}lP{6cm}@{}} 
    \textbf{Projektgebiet} & \textbf{Person} \\[5pt]
    App Backend &
      \begin{itemize}
        \item Björn 
        \item \dots
      \end{itemize}\\

    App Design &
     \begin{itemize}
         \item Sebi
         \item \dots
     \end{itemize}\\

    Web Server &
     \begin{itemize}
         \item \dots
     \end{itemize}\\
\end{tabular}

\section*{Aufgaben im Projekt}
\subsection*{App Backend}
\begin{itemize}
    \item Verbinden mit Web Server
    \item Verknüpfen einzelner Fragmente duch Buttons,\dots
    \item Hintergrundaktivität
\end{itemize}

\subsection*{App Design}
\begin{itemize}
    \item Erstellen eines Styler für Fragmente, Texte,\dots
    \item Festlegen eines Farb Themas in der App
    \item Evtl Style via CSS generieren
    \item Einbinden des Logos oder Festivalnamen als Überschriften
\end{itemize}

\subsubsection*{Web Server}
\begin{itemize}
    \item Aufsetzen eines Web Servers mittels Apache,\dots
    \item Anlegen der Daten für App in geeigneter Struktur
    \item Installieren und aufsetzen von RSS
\end{itemize}

\end{document}